\documentclass[11pt,a4paper,oneside]{report}
\usepackage{ngerman}
\usepackage[latin1]{inputenc}
\usepackage{latexsym,amssymb,mathtools,amstext}

\begin{document}

\begin{center}
  \Huge
   Versuch 1 \\[5mm]
   {\bf Pohlversuch}
\end{center}

\normalsize

\begin{center}
\begin{tabular}{|ll|ll|} \hline
  {\bf Nils Jacobsen:}  & Leon Plickat \\
  {\bf Duchgef�hrt am:} 23.11.2017\\ \hline
\end{tabular}
\end{center}


\section*{Einleitung}
Dem Pohl'schen Resonator liegt das physikalische Ph�nomen einer ged�mpften, mechanischen, harmonischen Schwingung zugrunde. 
Derartige Schwingungen spielen eine wichtige Rolle zum Verst�dnis mechanischer Grundlagen. Dar�ber hinaus lassen sich einige Erkenntnisse �ber mechanische Schwingungen auf andere Teilgebiete der Physik �bertragen, wenn es z.B. um Schwingungsvorg�ne in Atomen und Festk�rpern oder elektro-magnetische Schwingungen geht.
Der Pohlversuch soll zum Verst�ndnis von mechanischen Schwingungen i.A. beitragen und au�erdem im Umgang mit linearen Differentialgleichungen, Phasenverschiebungen und Resonanzerscheinungen schulen. 

\section*{Theorie}
Zun�chst wird die freie Schwingung also ohne Resonanzeinfl�sse betrachtet: 

Der Beschleunigung des Oszillators wirkt stehts die R�ckstellkraft entgegen, welche proportional zur Auslenkung wirkt. Auch Reibungseinfl�sse d�rfen nicht vernachl�ssigt werden. 
Diese k�nnen zum einen gezielt z.B. durch eine Wirbelstrombremse angesteuert werden, treten jedoch zwangsl�ufig wegen der Luftreibung auf. 
Somit ergibt sich die folgende lineare DGL:

\begin{align}
m \cdot \ddot{x} + \beta \cdot \dot{x} +  D \cdot x = 0
\end{align} 

In dieser Darstellung ist $m$ die Masse des schwingenden Systems, die idealisiert als punktf�rmig angenommen wird.
$\beta$ ist die D�mpfungskonstante der Schwingung und $D$ ist die Schwingungskonstante. 

Setzt man $\frac{D}{m} =\omega_0^2$ und $2\gamma = \frac{\beta}{m}$, ergibt sich f�r die DGL:

\begin{align}
\ddot{x} + 2\gamma \cdot \dot{x} + \omega_0^2 \cdot x = 0
\end{align}

Dann ist die L�sung mit dem Exponentialansatz:

\begin{align}
x(t)= e^{-\gamma \cdot t} \cdot \lbrack c_1 e^{\sqrt{\gamma^2-\omega_0 ^2}\cdot t} + c_2 e^{\sqrt{\gamma^2-\omega_0^2} \cdot t} \rbrack
\end{align}
Vgl. Demtr�der I S.335ff.

An dieser Stelle k�nnen drei verschiedene F�lle betrachtet werden.

%\[\gamma < \omega_0
  %\gamma > \omega_0
  %\gamma= \omega_0 \]  
  
In dem hier betrachteten Experiment ist der Fall $\gamma < \omega_0 $  relevant.

Setze $\omega_d^2=\omega_0^2-\gamma^2$
Dann ergibt sich f�r die Schwingungsgleichung die Funktion:

\begin{align}
 x(t) = x_0 e^{-\gamma \cdot t} \cdot \cos(\omega_d \cdot t + \phi)
\end{align}

Das Ergebnis ist also ein Kosinus-Schwingung deren Amplitude exponentiell abnimmt. 

Die Abnahme kann durch das logarithmische Dekrement beschrieben werden. 

Es gilt:

\begin{align}
\frac{x(t+T)}{x(t)} = e^{-\gamma \cdot T}
\end{align} 

Hierbei ist T die Dauer eines Schwingungsdruchgangs.

Das logarithmische Dekrement $\delta$ ist dann definiert als:

\begin{align}
\delta = \gamma\cdot T
\end{align}

F�r den  zweiten Teil des Experiments wird keine freie Schwingung mehr betrachtet, sondern der Fall, dass der Oszillator eine andere Schwingung angeregt wird. 

In diesem Fall wirkt zus�tzlich eine peridosiche Kraft auf das schwingende System der Form
$F=F_0 cos(\omega_e t)$. Dann ist die DGL:

\begin{align}
\ddot{x} + 2\gamma \cdot \dot{x} + \omega_0^2 \cdot x = K \cos(\omega_e t)
\end{align}

 (mit $K=\frac{F_0}{m} $)

Die L�sung ist dann:
\begin{align}
x(t) = x_0 e^{-\gamma \cdot t} \cdot \cos(sqrt{\omega_0^2 - \gamma^2}  \cdot t + \phi) + x_s \cdot \cos( \omega_e \cdot t + \phi_s)
\end{align} 

Der erste Kosinus Term ist der des Eigenschwingvorgangs und konvergiert wegen der reellwertigen Exponentialfunktion als Vorfaktor gegen $0$ f�r gro�e $t$ . Dann reicht es, den zweiten Kosinusterm zu betrachten. Dieser beschreibt die station�re Schwingung die durch den Errgeger erzeugt wird. $\omega_e$ ist die Errgerfrequenz. Die Phasenverschiebung $\phi_s$ und die Amplitude $x_s$ lassen sich dann durch bereits bekannte Konstanten ausdr�cken.

Es gilt: 

\begin{align}
x_s= \frac{K}{\sqrt{(\omega_0^2 -\omega_e^2)^2 +(\gamma \omega_e)^2}}
\end{align}
und 
\begin{align}
\phi_s= \arctan(\frac{2 \gamma \omega_e}{\omega_0^2 - \omega_e^2}
\end{align}

Es folgt, dass die Schwingungsamplitude von der Erregerfrequenz, der Eigenfrequenz des Schwingsystems und der D�mpfung abh�ngen.

Die Amplitude erreicht f�r die Resonanzfrequenz $\omega_r= \sqrt{\omega_0^2-2\gamma^2}$ das Maximum. Bei einer geringen D�mpfung ist $\omega_r \approx \omega_d$.













 
 

\section*{Durchf�hrung}


\section*{Auswertung}



\section*{Einordnung der Ergebnisse}

\section*{Literatur}
%\begin{thebibliography}{1}
%\bibitem{1} Wolfgang Demtr�der, \emph{Experimentalphysik 1 Mechanik und W�rme Lehre} S.328


\end{document} 
